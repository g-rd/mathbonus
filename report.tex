% --------------------------------------------------------------
% This is all preamble stuff that you don't have to worry about.
% Head down to where it says "Start here"
% --------------------------------------------------------------

\documentclass[12pt]{article}

\usepackage[margin=1in]{geometry}
\usepackage{amsmath,amsthm,amssymb}

\newcommand{\N}{\mathbb{N}}
\newcommand{\Z}{\mathbb{Z}}

\newenvironment{theorem}[2][Theorem]{\begin{trivlist}
\item[\hskip \labelsep {\bfseries #1}\hskip \labelsep {\bfseries #2.}]}{\end{trivlist}}
\newenvironment{lemma}[2][Lemma]{\begin{trivlist}
\item[\hskip \labelsep {\bfseries #1}\hskip \labelsep {\bfseries #2.}]}{\end{trivlist}}
\newenvironment{exercise}[2][Exercise]{\begin{trivlist}
\item[\hskip \labelsep {\bfseries #1}\hskip \labelsep {\bfseries #2.}]}{\end{trivlist}}
\newenvironment{reflection}[2][Reflection]{\begin{trivlist}
\item[\hskip \labelsep {\bfseries #1}\hskip \labelsep {\bfseries #2.}]}{\end{trivlist}}
\newenvironment{proposition}[2][Proposition]{\begin{trivlist}
\item[\hskip \labelsep {\bfseries #1}\hskip \labelsep {\bfseries #2.}]}{\end{trivlist}}
\newenvironment{corollary}[2][Corollary]{\begin{trivlist}
\item[\hskip \labelsep {\bfseries #1}\hskip \labelsep {\bfseries #2.}]}{\end{trivlist}}

\begin{document}

% --------------------------------------------------------------
%                         Start here
% --------------------------------------------------------------

%\renewcommand{\qedsymbol}{\filledbox}

\title{Homework}%replace X with the appropriate number
\author{Gerd Kukemilk\\ %replace with your name
Bonus homework task} %if necessary, replace with your course title

\maketitle


To find the modular multiplicative inverse we can use the extended Euclidean algorithm to find Bezout Identity \(a*s + x*t = 1\). This will give us Bezout coefficients s and t where \(t_{i-1}, \) where remainder is 0 is the modular multiplicative inverse if a in the integer for which we look for the inverse and b is the modulus. This works only if the GCD of a and x is 1, meaning that they are relatively prime.

There are two ways to calculate extended Euclidean Algorithm, the so-called backwards substitution, and the iterative way. Backwards substitution is best suited for calculating by hand while the other suits very well for calculating EEA programmatically.

To start, we note that we will calculate the coefficients without backwards substitution. For this we can express the EEA equation as follow:

\begin{align*}
a_1 = 1 * a_1 + 0*a_2 \\
a_2 = 0 * a_1 + 1*a_2 \\
a_i = s_i*a_1 + t_i*a_2 \\
a_{i+2} = a_i - q_i*a_{i+1} = \\
(a_1*s_i+a_2*t_i) - q_i(a_1*s_i+1 + a_2*t_{i+1}) = \\
a_1(s_i-q_i*s_{i+1}) + a_2(t_i-q_i*t_{i+1}) \\
\\
\end{align*}

Hence we get these equations that we use in our python code:
\begin{alignat*}{2}
    & \begin{aligned} & \begin{cases}
    s_1 = 1\\
    s_2 = 0\\
    s_{i+2} = s_i-q_i*s_{i+1}, i \geq 1 \\
  \end{cases}\\
  \end{aligned}
  \begin{aligned}
  & \begin{cases}
    t_1 = 0\\
    t_2 = 1\\
    t_{i+2} = t_i-q_i*t_{i+1}, i \geq 1 \\
  \end{cases} \\
  \end{aligned}
\end{alignat*}
\newpage
An example calculation of EEA where a = 3 and b = 8 (a is the integer and b is the modulus for multiplicative inverse)
\begin{table}[ht]
\begin{tabular}{|l|l|l|l|l|}
\hline
i & quotient     & remainder       &  \(s_i\)               & \(t_i\)               \\ \hline
1  &              & 8               & 1                  & 0                  \\ \hline
2  &              & 3               & 0                  & 1                  \\ \hline
3  & \(8 \div 3 = 2\) & \(8 - 2 * 3 = 2\) & \(1 - 2 * 0 = 1\)    & \(0 - 2 * 1 = -2\)   \\ \hline
4  & \(3 \div 2 = 1\) & \(3 - 1 * 2 = 1\) & \(0 - 1 * 1 = -1\)   & \(1 - 1 * (-2) = 3\) \\ \hline
5  & \(2 \div 1 = 2\) & \(2 - 2 * 1 = 0\) & \(1 - 2 * (-1) = 3\) & \(-2 * -2 * 3 = -8\) \\ \hline
\end{tabular}
\end{table}
% template author:
% * <arking4@g.coastal.edu> 2015-08-24T22:28:35.113Z:
%
%
%

\end{document}